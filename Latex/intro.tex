% !TEX root = Time_Evo.tex

化学变化是由反应体系内电子与原子核的运动和相互作用产生的,分子动力学的出现从微观上描述了化学反应的内部过程。基于第一性原理的理论计算为理论科学家的研究做出了长远的贡献。本论文主要讨论了如何用计算机程序模拟理论模型,将连续物理概念离散化处理等理论化学常规技术。关注的主要对象是求解含时薛定谔方程得到的波函数含时演化的性质,并尝试用简单的模型解释现实中的物理现象。理论计算的研究方法中最重要的两点性质包括算法的适应性和可移植性,所有算法耗费的计算资源都和体系的大小呈正相关关系,我们在相空间中研究并比较不同算法的适用性。由于计算资源和研究规模的限制,本课题选用的理论模型并不十分精确,因此主要是定性分析量子体系的性质;定量分析可以建立在正确的理论模型与更精确的数值方法的基础上。

近年来分子动力学快速发展,它的出现意味着物理化学研究从间接的宏观实验方法延伸到可以用理论模型模拟体系发生的过程。这样一来理论科学家们可以建立模型来研究孤立的基元反应,进而能对化学反应的微观过程有更深刻的认识。对基元反应的分离既可以从空间上做到,也可以从时间的维度完成。例如气固相表面的基元反应可以在高真空条件下将原子束轰击到表面上,可以视为空间上独立的反应;然而对于溶液中的基元反应,不能使用空间分离技术,可以通过超飞秒激光脉冲技术在时间尺度上分离反应。

在这里我们需要区分建立分子反应模型(modeling)和模拟分子反应体系(simulating)的概念。建模的目的是研究分子反应的微观机理,通过简单的计算验证一种可能的理论解释通常就足够了。而模拟反应的目的是定量地比较实验结果与理论预测情况。分子动力学模拟技术主要归功于计算机技术的发展和日益精确的实验研究方法。我们的课题主要涉及模拟反应体系,不过这一套理论工具同样适用于模型的建立。

基于第一性原理的主要量子力学理论方法有:
\begin{multicols}{2}
\begin{itemize}
  \item 经典轨迹法
  \item 半经典法
  \item 量子耦合法
  \item 含时量子方法
\end{itemize}
\end{multicols}
这些基本方法同时也是理论化学近似技术的理论基石。它们共同建立于波恩-奥本海默(Born-Oppenheimer)近似的假设上,即将原子核和电子的运动分开考虑。于是原子核的势能场就形成了电子运动和化学反应发生的势能面。




其中含时量子力学模型在分子动力学研究中有着重要的地位,其理论模型在计算机数值计算上易于实现性并且计算结果令人信服。含时模型的建立与发展基于含时薛定谔方程:
\begin{equation}
  i \hbar \frac{\partial \psi}{\partial t} = \hat{H} \psi  
\end{equation} 

从数值计算上来说,求解含时体系是一个初值问题,因此程序上容易实现。除此之外,其演化结果是直观可见的,可以有力地解释物理过程。通过引入含时哈密顿算符,含时算法还可以有效地描述受到外部作用的体系的演化情况,例如:体系在激光的作用下如何演化。本课题构建了一个一维简谐振子模型,用Chebyshev方法展开含时演化算符,研究其演化性质,并与其解析解进行比较。