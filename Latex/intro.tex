% !TEX root = Time_Evo.tex

化学变化是由反应体系内电子与原子核的运动和相互作用产生的,分子动力学的出现从微观上描述了化学反应的内部过程。基于第一性原理的量子力学理论模型有(a)经典轨迹模型(b)半经典模型(c)量子耦合模型(d)含时量子模型。

其中含时量子力学模型在分子动力学研究中有着重要的地位,其理论模型在计算机数值计算上易于实现性并且计算结果令人信服。含时模型的建立与发展基于含时薛定谔方程:
\begin{equation}
  i \hbar \frac{\partial \psi}{\partial t} = \hat{H} \psi  
\end{equation} 

从数值计算上来说,求解含时体系是一个初值问题,因此程序上容易实现。除此之外,其演化结果是直观可见的,可以有力地解释物理过程。通过引入含时哈密顿算符,含时算法还可以有效地描述受到外部作用的体系的演化情况,例如:体系在激光的作用下如何演化。本课题构建了一个一维简谐振子模型,用Chebyshev方法展开含时演化算符,研究其演化性质,并与其解析解进行比较。