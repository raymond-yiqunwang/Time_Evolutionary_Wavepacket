% !TEX root = Time_Evo.tex

化学变化是由反应体系内电子与原子核的运动和相互作用产生的,分子动力学的出现从微观上描述了化学反应的内部过程。基于第一性原理的理论计算为理论科学家的研究做出了长远的贡献。本论文主要讨论了如何用计算机程序模拟理论模型,将连续物理概念离散化处理等理论化学常规技术。关注的主要对象是求解含时薛定谔方程得到的波函数含时演化的性质,并尝试用简单的模型解释现实中的物理现象。理论计算的研究方法中最重要的两点性质包括算法的适应性和可移植性,所有算法耗费的计算资源都和体系的大小呈正相关关系,我们在相空间中研究并比较不同算法的适用性。由于计算资源和研究规模的限制,本课题选用的理论模型并不十分精确,因此主要是定性分析量子体系的性质;定量分析可以建立在正确的理论模型与更精确的数值方法的基础上。

近年来分子动力学快速发展,它的出现意味着物理化学研究从间接的宏观实验方法延伸到可以用理论模型模拟体系发生的过程。这样一来理论科学家们可以建立模型来研究孤立的基元反应,进而能对化学反应的微观过程有更深刻的认识。对基元反应的分离既可以从空间上做到,也可以从时间的维度完成。例如气固相表面的基元反应可以在高真空条件下将原子束轰击到表面上,可以视为空间上独立的反应;然而对于溶液中的基元反应,不能使用空间分离技术,可以通过超飞秒激光脉冲技术在时间尺度上分离反应。

在这里我们需要区分建立分子反应模型(modeling)和模拟分子反应体系(simulating)的概念。建模的目的是研究分子反应的微观机理,通过简单的计算验证一种可能的理论解释通常就足够了。而模拟反应的目的是定量地比较实验结果与理论预测情况。分子动力学模拟技术主要归功于计算机技术的发展和日益精确的实验研究方法。我们的课题主要涉及模拟反应体系,不过这一套理论工具同样适用于模型的建立。

基于第一性原理的主要量子力学理论方法有:
\begin{multicols}{2}
\begin{itemize}
  \item 经典轨迹法
  \item 半经典法
  \item 量子耦合法
  \item 含时量子方法
\end{itemize}
\end{multicols}
这些基本方法同时也是理论化学近似技术的理论基石。它们共同建立于波恩-奥本海默(Born-Oppenheimer)近似的假设上,即将原子核和电子的运动分开考虑。于是体系粒子产生的势能场就形成了电子运动和化学反应发生的势能面。下面我们简述这四种基本方法的原理和优缺点。

一般最常用的是经典轨迹法,该方法使用牛顿力学计算原子核在波恩-奥本海默势能面上的运动情况。当我们掌握势能面的信息和体系的初始状态后,不难根据牛顿力学定律预测其发展情况。经典轨迹法被广泛应用于描述基本的粒子碰撞问题,并且还结合拉格文公式(Langevin equation)拓展应用于凝聚态体系的研究。尽管如此,经典轨迹法也有致命的缺陷。对于一次精确的经典轨迹模拟,需要在相空间内大量采样,也将极大地增加计算资源消耗。这就造成处理缓慢发生的反应时的不便。更加重要的缺陷是经典方法没有考虑量子体系的量子效应,例如量子隧穿效应和零点能现象等。而对于微观的基元反应,这些量子效应有时是至关重要的,忽略它们可能造成模拟方法本质上的错误。

半经典法在沿用经典法牛顿力学体系的基础上做了量子效应的相关修正。这种处理方法的合理性源于原子体系中原子核的质量和体积相对很大,电子等小粒子的量子效应可以看做是对经典模型的修正项,也就是说体系的基本运动符合牛顿力学的描述。开发最完善的半经典方法是高斯波包近似。该近似将体系的量子波包限定为高斯波包的形式,这一方法在很多体系的应用中取得不小的成功。然而半经典法是一种较微妙的方法,对于每一种体系需要做考虑其独特的量子效应,正如之前所说,其适用性和可移植性不能得到保证。另外,高斯波包近似的稳定性也并没有得到严谨的证明。

之所以我们尚且无法做到非常精准的量子体系计算,主要是由量子体系的非定域性导致的,也就是说我们需要在相空间中同时表示出全局的信息。相对较为成熟的量子计算方法是基于求解静态散射方程得到的静态耦合通道方法。该方法被应用于解决许多动力学问题,不过其发展进步相当缓慢。从$H + H_2$体系发展到$D + H_2$花费了科学家10年时间。主要问题在于该算法的复杂度为$O(N^3)$,占用很多计算资源。另外还有边界性问题等也阻遏着量子耦合法的发展。

含时量子力学模型在分子动力学研究中有着重要的地位,其建立与发展基于求解含时薛定谔方程:
\begin{equation}
  i \hbar \frac{\partial \psi}{\partial t} = \hat{H} \psi  
\end{equation} 
含时方法在处理很多物理问题时有其独特的优势。由于求解含时体系是一个初值问题,该方法在程序上易于实现;含时方法在解释物理现象时形象易懂,我们很容易从时间推移的角度理解反应的进程;另外,通过引入含时哈密顿量,含时方法还可以描述外部环境导致的系统演化历程,例如:体系在激光的作用下如何演化。精确的含时演化方法发展较为缓慢。期初通过利用基函数展开体系的波函数并对角化哈密顿矩阵得到特征向量和本征值,再进一步构造含时演化算符进行演化。矩阵对角化的算法复杂度为$O(N^3)$,这对于处理较大的真实体系造成相当大的困难。然而快速傅里叶变换(FFT)算法的出现有效地解决了这一困难,它将算法复杂度降低到$O(N\log N)$,在现代计算机体系结构下可以进一步优化到$O(\log N)$。傅里叶方法的准确性和高效性为含时方法的发展做出了巨大的贡献。

本课题构建了一个一维简谐振子模型,用Chebyshev方法展开含时演化算符,研究其演化性质,并与其他含时演化方法进行比较。