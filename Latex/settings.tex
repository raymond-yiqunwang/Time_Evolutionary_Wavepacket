% !TEX root = Time_Evo.tex


\usepackage[retainorgcmds]{IEEEtrantools}
\usepackage{array}
\usepackage{multicol}
\usepackage{indentfirst}
\usepackage[font=small,labelfont=bf]{caption}
\setlength\parindent{0.85cm}
\setlength{\columnsep}{0.7cm}
\setlength{\parskip}{5pt}
\usepackage{pdfpages}
\usepackage{xcolor,graphicx}
\usepackage[hmargin=2.2cm,vmargin=2.5cm]{geometry}
\usepackage{makeidx}
\usepackage{amsmath,amsthm}
\usepackage[retainorgcmds]{IEEEtrantools}
\renewcommand\refname{参考文献}
\renewcommand\abstractname{摘要}
\renewcommand\tablename{表}
\renewcommand\figurename{图}
\newcolumntype{P}[1]{>{\centering\arraybackslash}p{#1}}
\makeindex

%-----------------------------CJK Font Setup------------------------------!

\usepackage[PunctStyle=kaiming,AutoFakeBold=false,AutoFakeSlant=false,indentfirst,CJKnumber]{xeCJK}
\xeCJKDeclareSubCJKBlock{PUNC}{"FF1F}
\setCJKmainfont[BoldFont={Songti SC Bold}, PUNC=SimSun] {Songti SC Light}
\setCJKsansfont[BoldFont={Heiti SC Medium}]{Heiti SC}
\setCJKmonofont[BoldFont={Hiragino Sans GB W6}]{Hiragino Sans GB W3}

\setCJKfamilyfont{zhxinsong}{SimSun}
\setCJKfamilyfont{zhsong}[BoldFont={Songti SC Bold}]{Songti SC Light}
\setCJKfamilyfont{zhhei}[BoldFont={Heiti SC Medium}]{Heiti SC}
\setCJKfamilyfont{zhfs}{FangSong}
\setCJKfamilyfont{zhkai}[BoldFont={Kaiti SC Bold}]{Kaiti SC Regular}
\setCJKfamilyfont{zhli}{LiSu}
\setCJKfamilyfont{zhyou}{YouYuan}

\newcommand*{\simsun}{\CJKfamily{zhxinsong}} % 新宋
\newcommand*{\songti}{\CJKfamily{zhsong}}    % 宋体
\newcommand*{\heiti}{\CJKfamily{zhhei}}      % 黑体
\newcommand*{\fangsong}{\CJKfamily{zhfs}}    % 仿宋
\newcommand*{\kaishu}{\CJKfamily{zhkai}}     % 楷书
\newcommand*{\lishu}{\CJKfamily{zhli}}       % 隶书
\newcommand*{\youyuan}{\CJKfamily{zhyou}}    % 幼圆

% Use Fontsize inside {}, include a \par at the end of bracket to enable baselineskip
\newcommand{\chuhao}{\fontsize{42pt}{50.4pt}\selectfont}       %初号
\newcommand{\xiaochu}{\fontsize{36pt}{43.12pt}\selectfont}     %小初号
\newcommand{\yihao}{\fontsize{27.5pt}{33pt}\selectfont}        %一号
\newcommand{\erhao}{\fontsize{21pt}{25.2pt}\selectfont}        %二号
\newcommand{\xiaoer}{\fontsize{18pt}{21.6pt}\selectfont}       %小二号
\newcommand{\sanhao}{\fontsize{15.75pt}{18.9pt}\selectfont}    %三号
\newcommand{\sihao}{\fontsize{13.75pt}{16.5pt}\selectfont}     %四号
\newcommand{\xiaosi}{\fontsize{12pt}{14.4pt}\selectfont}       %小四号
\newcommand{\wuhao}{\fontsize{10.5pt}{12.6pt}\selectfont}      %五号
\newcommand{\xiaowu}{\fontsize{9pt}{10.8pt}\selectfont}        %小五号
\newcommand{\liuhao}{\fontsize{7.875pt}{9.45pt}\selectfont}    %六号
\newcommand{\qihao}{\fontsize{5.25pt}{6.3pt}\selectfont}       %七号

%----------------------------English Font Setup---------------------------!
\usepackage{fontspec,xltxtra,xunicode}
\defaultfontfeatures{Mapping=tex-text}
\setmainfont{Times New Roman}
\setsansfont[Scale=MatchLowercase,Mapping=tex-text]{Helvetica}
\setmonofont[Scale=MatchLowercase]{Times New Roman}

\newfontfamily{\AMT}{American Typewriter}
\newfontfamily{\Curly}{Baskerville}
\newfontfamily{\Fraktur}{Euclid Fraktur}


%Define Han Length
\newlength{\Han}
\settowidth{\Han}{汉}

%-------------------------------Language Setup----------------------------!

%-------------------------------Special Attention-------------------------!
\newsavebox{\myhbar}
\savebox{\myhbar}{$\hbar$}
\renewcommand*{\hbar}{\mathalpha{\usebox{\myhbar}}}
