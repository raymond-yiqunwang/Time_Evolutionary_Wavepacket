% !TEX root = Time_Evo.tex

\subsubsection{含时演化结果演示}
在量子力学中,量子系统的状态用波函数(wave function)来描述,在这里我们使用$\Psi(r,t)$表示被研究体系的总波函数,$r$是体系的位置,$t$为时间。求解含时薛定谔方程可以得到波函数随时间演化的所有信息,波函数是波动方程的解,具有类波的性质。需要注意,波函数$\Psi(r,t)$本身是复值函数,其值表示粒子在某一确定的时间空间点处的概率幅度;而波函数表达式的模平方${|\Psi(r,t)|}^2$表示在该时空点上粒子出现的概率,因此量子体系并不像经典力学体系那样具有确定性,它是建立在统计力学之上的。

为了将波函数概率分布变得可视化,我们将${\Psi(r,t)}^2$对空间坐标量对应作图。由于哈密顿量选取的是一维简谐振子模型,我们将初始的波函数设为一系列分立的振动能级本征波函数,求解薛定谔方程:
\begin{equation}
  -\frac{\hbar^2}{2\mu}\frac{d^2\Psi}{dx^2} + \frac{1}{2}kx^2\Psi = E \Psi
\end{equation}
\noindent 得到能量本征值为:
\begin{equation}
  E_{\nu} = \left( \nu + \frac{1}{2}\right ) h \nu_e
\end{equation}
\noindent 其中$\nu$为整数$0,1,2,\cdots$,相应的本征函数为:
\begin{equation}
  \Psi = N_{\nu}e^{-q^2/2}H_{\nu}(q)
\end{equation}
\noindent 其中$N_{\nu} = {\left[ \sqrt{b/\pi} / 2^{\nu} \nu ! \right]}^{1/2}$是归一化系数,$H_{\nu}$是厄米多项式,
\begin{equation}
  H_{\nu}(q) = \nu ! \sum_{k=0}^{\nu / 2} \left[ \frac{{(-1)}^k}{k! {(\nu - 2k)!}} \right] {(2q)}^{\nu-2k}
\end{equation}
可以得到如图\ref{fig:Psi_Fig}的线性谐振子能级、波函数和概率分布图。
\begin{figure}[h]
  \center
  \vspace{-1mm}
  \includegraphics[width=0.9\linewidth]{Psi_Fig.png}
  \caption{线性谐振子能级、波函数和概率分布图}
  \label{fig:Psi_Fig}
\end{figure}

正如\ref{sec:basis}所述,我们使用一系列$sinc$函数拟合谐振子的波函数。首先声明程序中使用的默认参数:全部使用原子单位(a.u.);采样点数为100并等距采样,采样空间为[-2,2];简谐振子的力常数$k=4{\pi}^2{\nu_e}^2\mu$中,令$\nu_e$和$\mu$均为1;含时演化的单位时间间隔$dt$定为0.01。

将$\Psi$用$sinc$函数展开后得到一个系数向量,在接下来的处理中这个向量将代替波函数的具体形式出现,即所有算符操作都是作用在系数向量上的。在构造含时演化算符$U(t) = e^{(-i/\hbar)\hat{H}t}$在希尔伯特空间中的表示矩阵后,每一步含时演化即将演化算符作用在系数矩阵上。将作用后的新系数向量归一化并做逆变换,就得到了单步演化后的波函数。由于我们采用间断演化,时间尺度上是不连续的,因此$dt$的取值越小则越接近真实的情况;由于原子单位中单位时间约为2.4$\times$10$^{-17}$s,可以认为我们的时间间隔足够小,能够大约模拟出含时演化的过程。

按照理论预测,我们构造的是一维简谐振子的哈密顿矩阵,当相应的含时演化算符作用在一维谐振子本征态的时候,应该预期得到不随时间变化的波函数;而当演化算符作用在非本征态时,波函数的空间概率分布将会发生变化。我们的计算机程序给出了这些过程的演化情况:
\begin{figure}[hbt]
  \center
  \vspace{1mm}
  \includegraphics[width=0.75\linewidth]{v=0}
  \includegraphics[width=0.75\linewidth]{v=1}
  \caption{简谐振子本征态含时演化示意图\\
            (波函数表达式) \label{fig:prop1}}
\end{figure}

\begin{figure}[hbt]
  \center
  \vspace{1mm}
  \includegraphics[width=0.75\linewidth]{v=0}
  \includegraphics[width=0.75\linewidth]{v=1}
  \caption{非本征态含时演化示意图 \label{fig:prop2}}
\end{figure}

\subsection{切比雪夫展开与对角化方法比较}
\subsection{调整参数对演化的影响} 