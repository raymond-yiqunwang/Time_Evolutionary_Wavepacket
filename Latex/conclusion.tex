% !TEX root = Time_Evo.tex

本课题目前成功构建了以一维简谐振子模型为基础的含时演化算符,该算符适用于研究从给定的简谐振子初始状态到以后任意时刻的体系状态波函数演化的情况。在这里虽然我们使用的是简单的$sinc$函数作为基函数,并且在计算哈密顿量的数值方法中使用了三点近似和局部势能近似,但是从计算结果看来与理论预测相符。我们的目的并不是定量地准确研究体系的演化信息,而是尝试构造这样一个系统并用计算机程序从数值上实现它。从最终的程序运行结果来看本课题的模型构建较为成功,得到了我们想要的结果。

虽然本课题的量子化学理论背景并不深奥,主要涉及量子力学量在希尔伯特空间中的表示和含时演化算符的数值实现方法,但是理论学习和实际程序化是有差异的。在具体的构造过程中我们遇到了很多困难,例如:选取基函数时,既要满足能足够准确地展开波函数,又要满足正交归一的性质,还需要考虑边界条件和数值积分的稳定性。为了简易起见我们选用了$sinc$函数作为基函数,但是显然使用傅里叶级数的数值稳定性和算法效率都会更高,下一步我们将改用傅里叶级数为基函数。另外我们认为计算的误差来源还来自计算动能项中使用的三点近似方法,如果利用傅里叶变换理论上可以获得更高的精度,我们也考虑进一步优化动能项的算法。还有一方面的困难来自程序化方面,数值方法很多时候并不像理论那样可以给出很优美的解,数值计算可能会遇到一些意想不到的困难。

我们主要的研究对象--含时演化算符是含时方法的关键,例如我们想要研究关联函数$\left< \xi(0) | \hat{O} | \Psi(t') \right>$在任意时刻$t'$的信息,我们就可以将其展开为:
\begin{equation*}
  \left< \xi(0) \right| \hat{O} \left| \Psi(t') \right> \approx e^{\beta} \sum_{n=0}^N a_n \phi_n \left< \xi(0)\right| \hat{O} \left| \Psi(0) \right>
\end{equation*}
其中最常见的是偶极矩算符$\hat{\mu}$,通过对这些关联函数进行傅里叶变换我们可以模拟拉曼光谱等\cite{Raman}。如果条件允许我们也将进行光谱的模拟计算。

得益于计算机技术的高速发展,例如大规模并行化和量子计算机的问世,理论化学家可以研究的化学体系越来越复杂。含时量子力学方法在高性能计算的辅助下蓬勃发展,其高准确性、可移植性和高效性使它成为未来研究反应动力学和量子化学的利器。本课题模型的构建让我们得以管窥理论化学的宏伟体系。