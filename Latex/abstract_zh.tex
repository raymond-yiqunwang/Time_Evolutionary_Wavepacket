% !TEX root = Time_Evo.tex

本课题构造了含时量子力学方法中使用的含时演化算符,其中使用了一维简谐振子的哈密顿量,并研究了不同波函数在该含时演化算符作用下的演化情况。我们使用$sinc$函数为基函数展开研究体系的波函数,并根据基函数的性质构造了一维简谐振子的哈密顿量在希尔伯特空间中的表示矩阵。切比雪夫展开法和对角化方法分别被用来展开含时演化算符,我们观察了不同的始态波函数的演化情况,对比了两种展开方法的数值表现。结果发现在我们所研究的小体系中切比雪夫展开和对角化方法在数值的表现几乎一致,因此在一定范围内可以相互验证彼此的稳定性与准确性。我们还研究了含时演化算符展开项表达式中的影响因素,如采样格点数,切比雪夫展开项数和时间步长等。目前看来含时演化算符的构造较为成功,下一步是优化所选用的基函数,改进哈密顿量中使用的近似方法,并利用含时演化算符模拟简单的拉曼光谱。